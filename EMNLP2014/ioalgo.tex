\documentclass[twoside,11pt]{article}
%\usepackage{jair, theapa, rawfonts}

\usepackage{latexsym}
\usepackage{amsmath}
\usepackage{multirow}
\usepackage{url}
\DeclareMathOperator*{\argmax}{arg\,max}
%\setlength\titlebox{6.5cm}    % Expanding the titlebox

\usepackage{qtree}
%\setlength\titlebox{6.5cm}    % Expanding the titlebox



\usepackage[noend]{algorithmic}
\usepackage{algorithm}
\usepackage{xspace}
\usepackage{amsmath}

%\newcommand{\tikzset}[1]{}
\newcommand{\e}[1]{\hat{#1}}

\newcommand{\commentout}[1]{}

\newcommand{\shorten}[1]{}

\newcommand{\tcommentout}[1]{#1}
%\newcommand{\tcommentout}[1]{}



\newfont{\msym}{msbm10}
\newcommand{\reals}{\mbox{\msym R}}
\newcommand{\qed}{{\setlength{\fboxsep}{0pt}
    \framebox[7pt]{\rule{0pt}{7pt}}}}
\newcommand{\newcite}{\citeA}

\newtheorem{theorem}{Theorem}
\newtheorem{definition}{Definition}
\newtheorem{remark}{Remark}
\newtheorem{proposition}{Proposition}
\newtheorem{condition}{Condition}
\newtheorem{lemma}{Lemma}
\newtheorem{opt}{Optimization Problem}
\newtheorem{assumption}{Assumption}

\date{}

%\twocolumn


\newcommand{\rione}{r^{(i,1)}}
\newcommand{\ritwo}{r^{(i,2)}}
\newcommand{\rithree}{r^{(i,3)}}
\newcommand{\xione}{t^{(i,1)}}
\newcommand{\xitwo}{t^{(i,2)}}
\newcommand{\xithree}{t^{(i,3)}}
\newcommand{\aione}{a_i}
\newcommand{\aitwo}{a^{(i,2)}}
\newcommand{\aithree}{a^{(i,3)}}
\newcommand{\yione}{y^{(i,1)}}
\newcommand{\yitwo}{y^{(i,2)}}
\newcommand{\yithree}{y^{(i,3)}}
\newcommand{\phii}{\phi^{(i)}}
\newcommand{\bi}{z^{(i)}}
\newcommand{\oi}{o^{(i)}}

%\pagestyle{plain}

\begin{document}
%\maketitle
%\maketitle

\newcommand{\p}{{\cal P}}
\newcommand{\internal}{{\cal I}}
\newcommand{\n}{{\cal N}}
\newcommand{\rules}{{\cal R}}
\newcommand{\srule}{{a \rightarrow b \; c}}
\newcommand{\pa}{\mathrm{pa}}
\newcommand{\lc}{\mathrm{lc}}
\newcommand{\rc}{\mathrm{rc}}
\newcommand{\diag}{\mathrm{diag}}
\newcommand{\tleft}{\beta}
\newcommand{\tright}{\gamma}
\newcommand{\tree}{\tau}



\begin{figure}[h]
%\begin{footnotesize}
\framebox{\parbox{\columnwidth}{
{\bf Inputs:} Sentence $f_1 \ldots f_N$, 
L-SCFG $(\n, S, m, n)$, 
parameters
$C_r \in
\reals^{(m \times m \times m)}$, $\in \reals^{(m \times m)}$, or $\in
\reals^{(1 \times m)}$ for all $r \in \rules$,
$c^\infty_{X \rightarrow x} \in \reals^{(1 \times m)}$
for all $x \in [n]$,
$c^1_S \in \reals^{(m \times 1)}$.  


{\bf Data structures:} 
\begin{itemize}

\item Each $\alpha^{X, i, j} \in \reals^{1 \times m}$,
$1 \leq i \leq j \leq N$ is a row vector of inside terms.

\item Each $\beta^{X, i, j} \in \reals^{m \times 1}$,
$1 \leq i \leq j \leq N$ is a column vector of outside terms.

\item Each $\mu(X, i, j) \in \reals$
$1 \leq i \leq j \leq N$ is a marginal probability.

\end{itemize}

{\bf Algorithm:}

(Inside base case)
$
\forall i \in [N], \;\;
\alpha^{X, i, i} 
= \sum_{r \in X \rightarrow f_1} c^\infty_{r} 
$

(Inside recursion)
For $1 \leq i < j \leq N, \;\;
$
\begin{align*}
\alpha^{X, i, j} &=
\sum_{r \in X \rightarrow f_1 X_1 f_2 X_2 f_3}  \sum_{k=i}^{j-1} 
C_r(\alpha^{X, i + |f_1|, k}, \alpha^{X, k+1+|f_2|, j-|f_3|}) + \\
&\sum_{r \in X \rightarrow f_1 X_1 f_2} C_r (\alpha^{X,
    i+|f_1|,j-|f_2|}) + \sum_{r \in X \rightarrow f_1} C^\infty_r
\end{align*}

(Outside base case)
$
\beta^{S, 1, n} = c^1_S
$

(Outside recursion)
For $1 \leq i \leq j \leq N, \;\;
$
\begin{align*}
\beta^{X, i, j} &=
\sum_{r \in X_1 \rightarrow f_1 X_2 f_2 X f_3} 
\left (\sum_{k=1}^{i-1} 
C^{(1,2)}_{r}(\beta^{X, k, j}, \alpha^{X, k, i-1}) 
 + \sum_{k=j+1}^{N} 
C_{r}^{(1,3)}(\beta^{X, i, k}, \alpha^{X, j+1, k})
\right) \\
& + \sum_{r\in X_1 \rightarrow f_1 X f_2} \left( \sum_{k=1}^{i-1} \beta^{X, k,
    j} C_r + \sum_{k=j+1}^N \beta^{X, i, k} C_r
\right )
\end{align*}
\hbox{(Marginals)}
$
1 \leq i \leq j \leq N,\;\;
$
\[
\mu(X, i, j) = \alpha^{X, i, j} \beta^{X, i, j}
= \sum_{h \in [m]} \alpha^{X, i, j}_h \beta^{X, i, j}_h
\]
}}
\caption{\small The tensor form of the inside-outside algorithm, for
  calculation of marginal terms $\mu(X, i, j)$.  These marginals correspond to non-terminal span marginals; a slight modification in the marginal computation occurs for rule marginals.} 
\label{fig:spio2}
\end{figure}


\end{document}